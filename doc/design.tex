\documentclass[a4paper]{article}
\usepackage{a4wide}
\usepackage{latexsym, rslenv}

\title{RSL Type checker Version 1.1\\Design notes}
\author{Chris George, UNU/IIST}
\date{22 May 1998}
\begin{document}
\maketitle


\section{Introduction}
\label{sec:introduction}

This document provides some basic notes on the design of the RSL Type
Checker, intended to be of use to people interested in extending the
tool.


\section{Environment}
\label{sec:environment}

The static environment acts like a kind of structured symbol table.
There is a class environment (CLASS\_ENV) for each class expression.

For a basic class expression this has 8 components:
\begin{enumerate}
\item types
\item values
\item variables
\item channels
\item modules
\item axioms
\item with
\item adapts
\end{enumerate}

The first 6 hold information about entities of each kind declared in
the class expression.  (The 5th is called modules and capable of
holding schemes as well as objects, although schemes within class
expressions are not allowed.  The same structure is used for the
global environment, where schemes may occur.)

The 7th holds information about with clauses as a list of object
expressions.  With clauses are inherited by inner environments by
being copied when they are created.

The 8th is formed from renamings and hides.  Only Id\_or\_ops are
recorded as disambiguations on hides and renamings are ignored.

Except for axioms, There is an entity\_id (Type\_id, etc.) for each
entity.  These are effectively references into a table for each kind
of entity.  These tables hold additional information about the entity,
such as its identifier and its type or class.  The environment
component is just a list of the entity\_ids of each kind defined in
the class.  For axioms we just use a list of identifiers, the axiom
names, as there is no additional information to keep about axioms.

There is therefore eventually an entity\_id for each declared entity
and in many cases we \emph{normalise} applied occurrences of names
into references to their entity\_ids --- so that we don't have to keep
searching for them.  Such searching is called \emph{lookup}.  

For example, for type expressions, there is in the abstract syntax
both named\_type, which is a NAME, and defined\_type which is a
Type\_id.  Normalising a type converts the former into the latter.
There is also both a function type and a normalised version fun in which
accesses are normalised.  

Normalising is not currently done for value names --- this may have to
be added later for some extensions.

Apart from values, each of the first six environments is just a list
of entity\_ids.  For values, we have inner environments caused by the
declarations in quantifiers, let expressions etc.\ and so we need a
list of lists (where the inner lists are used as stacks, innermost
scope at the top).

No attempt is currently made to make environments more efficiently searched by
making them ordered (by their identifiers), or by using ordered trees
instead of lists. 

As well as the basic environment there is an extend environment for
dealing with extending class expressions.  (All others become basic
environments.)  These are needed because the two components of an
extend may have different withs and adapts, and because names
occurring in the second class of an extend may be defined in the
first, but not vice versa.  

Associated with an environment is a PATH to keep track of where we are
in extending class expressions.  This is just a list of ``left'' and
``right''.  For example, if we are dealing with the class
expression

\kw{extend} \kw{extend} CE1 \kw{with} \kw{extend} CE2 \kw{with} CE3 \kw{with} CE4

then the paths in force when we are dealing with each of the
constituent class expressions will be as follows

\begin{tabbing}
  CE1 \= left(left(nil)) \\
  CE2 \> left(right(left(nil))) \\
  CE3 \> left(right(right(nil))) \\
  CE4 \> right(nil) 
\end{tabbing}

\subsection{Current environment}
\label{sec:current_env}

The current environment is held in the global variable ``Current''.
It is a list (stack) of environments.  The stack reflects nesting of
environments caused by object definitions and local expressions.  The
innermost environment is at the top of the stack (front of the list).

There is also a stack of PATHs held in the global variable
``Extend\_paths''.  It is important that Current and Extend\_paths
always have the same depth.

\subsection{Global environment}

The context of the module being checked is stored in the global
variable ``Globals''.  It has the same structure as the modules
component of a class environment.

\subsection{Formal scheme parameters}

When checking a parameterised scheme the global variable
``Parameters'' holds the formal parameters, again a module environment
although only objects will appear in it.


\section{Type checking algorithm}
\label{sec:algorithm}

The overall algorithm for checking the ``top level'' module is
\begin{enumerate}
\item Check the context modules and store their environments in
  Globals.
\item (For a parameterised scheme) check the formal parameters and store their environments in
  Parameters.
\item Check the body
\end{enumerate}

In each case ``Check'' means, after parsing and creating the abstract
syntax tree, do four passes.
\begin{enumerate}
\item Create the class environment structure and the entity\_ids for object,
  type, variable and channel definitions.  
\item Complete the type environment by normalising abbreviation types
  and storing them in the Type\_ids.
\item Add the value environment, with normalised types.  Also adds
  types to variable\_ids and channel\_ids, and adds the axiom
  environment.  Also adds value definitions implicit in variant,
  record and union types.
\item Type check value expressions
\end{enumerate}

Multiple passes are due to the lack of a ``define before use'' rule in
RSL.  We need first to establish a type and object environment, which
takes two passes, and then two more to establish value types and
accesses and then do type-checking.


\section{Looking up names}
\label{sec:lookup}

There are separate functions for objects, types, values, variables and
channels but they all follow the same pattern.  (I think it would need
parametric polymorphism in Gentle, or the use of only one kind of
entity\_id, to use the same code for them.)  The strategy is

\begin{enumerate}
\item Lookup name in current environment, moving if necessary left through current
  path (for extend) and then looking in outer scopes.
\item For each environment, if we came from outside the environment
  (e.g.\ we are looking in the environment of X for a name X.N) we
  have to adapt the name according to the adapts of that
  environment).  The parameter of type OWNENV is used to indicate
  looking from inside (ownenv) or from outside (nil).
\item For each environment we have to generate a set of possible names
  if there are with clauses.  E.g.\ in \kw{with} A \kw{with} B ... we
  need to look for N, A.N, B.N and A.B.N
\item For a qualified name X.N we have to obtain the environment of X
  and then lookup N within it.
\item For objects (and this includes the qualifiers of qualified
  names), if the object or qualifier is not defined in the current
  environment we normally move out to the enclosing environment.  But
  if the current environment is the result of unfolding a scheme
  instantiation we must ``fit'' the object name using the fitting
  found in the current environment.  This relates formal to actual
  parameters (including fittings) when we are checking a scheme
  instantiation.  E.g.\ if we are looking for X.N and we find X is a
  formal parameter for which the corresponding actual parameter is
  \mbox{A{\LBRACE}N{\PRIM} \kw{for} N{\RBRACE}} then we need to lookup
  A.N{\PRIM}.  If X is not a formal parameter (i.e.\ is not found in
  the fitting), then we next look in the global environment, since this
  is the only place an object name occurring in a scheme but not
  defined inside it or in its formal parameters could have been
  defined.
\end{enumerate}

For objects, types, variables and channels we use the first we find
(as inner scopes hide outer ones).  For values, as we have
overloading, we search completely and collect a set of possible types.

\section{Known shortcomings}
\label{sec:shortcomings}

\begin{enumerate}
  \item Objects used as qualifiers in type expressions are not type
    checked.  This needs to be done as part of the final pass, since
    it depends on the value environment.
  \item Subtypes need to be checked after the value enviroment is
    created.  This is done with Gentle's sweep facility, which will
    fail if there are local expressions within a subtype.  Strange but
    possible!
  \item Check\_type\_defined is used to normalise types in some places
    during type checking.  This can duplicate error messages.

\end{enumerate}
\end{document}
